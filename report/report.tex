%%%%%%%%%%%%%%%%%%%%%%%%%%%%%%%%%%%%%%%%%%%%%%%%%%%%%%%%%%%%%%%%%
\documentclass[russian]{article}
\usepackage[utf8]{inputenc}
\usepackage[T1]{fontenc}
\usepackage{babel}

\usepackage{pscyr} % улучшенный кириллический шрифт
% пакеты amssymb и amsmath рекомендуется подключать всегда,
% если в вашем тексте присутствуют сколько-нибудь сложные формулы
\usepackage{amssymb}
\usepackage{amsmath}
\usepackage{amsfonts} % подкл. дополнительных мат. шрифтов
\usepackage{latexsym} % некоторые редкие символы
% \usepackage[dvips]{graphicx} % импорт графических изображений в LaTeX
\usepackage[pdftex]{graphicx} % импорт графических изображений в PdfLaTeX
\usepackage{geometry} %
\usepackage{indentfirst} % отступ в первом абзаце
\usepackage{cite} % ссылки на литературу в виде [1-5,9]
% \usepackage{listings}
\usepackage{fancyvrb} % улучшенный вариант verbatim
%\usepackage{shortvrb) % сокращенный вариант verbatim
% \usepackage{verbatim}
\usepackage[large]{caption2}
\usepackage{url}

\pagestyle{myheadings} % верхний колонтитул (номер страницы)

\newcommand{\imsize}{0.45\columnwidth}
\newcommand{\bukva}[1]{\begin{tabular}[b]{c} (#1)\\ \\ \\ \end{tabular}}

%Изменение значения междустрочного интервала для всего документа
%\linespread{1.5}

\geometry{verbose,a4paper,tmargin=2.5cm,bmargin=2.5cm,lmargin=2.5cm,rmargin=1.0cm}

\begin{document}
\section{Цели и задачи}

\subsection{Цель работы}

Целью лабораторной работы является практическое освоение информации,
полученной при изучении курса «Компьютерное моделирование в
математической физике» по теме «Численное решение стационарного
уравнения Шрёдингера», а также развития алгоритмического мышления и
приобретения опыта использования знаний и навыков по математике,
численным методам и программированию для решения прикладных задач
физико--технического характера.

\subsection{Задачи работы}

\textbf{Проблема}: электрон находится в одномерной потенциальной яме с
бесконечными стенками

\[v(x) = \begin{cases}
0\text{, } & \begin{aligned}
\text{    } & x \in \left\lbrack - \frac{L}{2},0 \right\rbrack
\end{aligned} \\
 - 1\text{, } & \begin{aligned}
\text{    } & x \in \left\lbrack - L, - \frac{L}{2} \right\rbrack\forall x \in \lbrack 0,L\rbrack
\end{aligned} \\
\infty\text{, } & \begin{aligned}
\text{    } & x \leq - L\forall x \geq \lbrack 0,L\rbrack
\end{aligned}
\end{cases}\] где \(V_{0} = 20эВ,L = 2A\).

Поставлена задача, используя метод пристрелки, найти энергию,
нормированную волновую функцию и плотность вероятности для основного и
1-го возбужденного состояний. После проведенных рассчетов требуется для
каждого из вычисленных состояний найти квантовомеханические средние
\(\langle x\rangle\) и \(\langle x^{2}\rangle\).

\section{Одномерное стационарное уравнение Шрёдингера. Математический
формализм. Общие свойства решений.}

Одномерное стационарное уравнение Шрёдингера

\[\hat{H}\psi(x) = E\psi(x)\]

с математической точки зрения представляет собой задачу определения
собственных значений \(E\) и собственных функций \(\psi(x)\) оператора
Гамильтона \(\hat{H}\). Для частицы с массой m, находящейся в
потенциальном поле (называемом также потенциалом) \(U(x)\), оператор
Гамильтона имеет вид

\[\hat{H} = \hat{T} + U(x)\]

Собственное значение оператора Гамильтона имеет смысл энергии
соответствующей изолированный квантовой системы. Собственные функции
называются волновыми функциями. Волновая функция однозначна и непрерывна
во всем пространстве. Непрерывность волновой функции и ее 1-ой
производной сохраняется и при обращении \(U(x)\) в \(\infty\) в
некоторой области пространства. В такую область частицы вообще не может
проникнуть, то есть в этой области, а также на её границе (что следует
из непрерывности функции), \(\psi(x) = 0\).

Оценим нижнюю границу энергетического спектра. Пусть минимальное
значение потенциальной энергии равно \(U_{\min}\). Очевидно, что
\(\langle T\rangle \geq 0\) и \(\langle U\rangle \geq U_{\min}\).
Потому, из уравнения (2) следует, что

\[E = \langle H\rangle = \int\limits_{- \infty}^{+ \infty}\psi^{\ast (x)}\hat{H}\psi(x)dx = \langle T\rangle + \langle U\rangle > U_{\min}\]

Особый практический интерес представляет случай, когда

\[\lim\limits_{x \rightarrow \pm \infty}U(x) = 0\]

Потенциал такого типа (называемый также потенциальной ямой) изображен на
рис. 1. Для данной \(U(x)\) свойства решений уравнения Шрёдингера
зависят от знака собственного значения \(E\).

Рассмотрим случай, когда \(E < 0\). Частица с отрицательной энергией
совершает финитное движение. Оператор Гамильтона имеет дискретный
спектр, то есть собственные значения и соответствующие собственные
функции можно снабдить номерами (называемые квантовыми числами). При
\(E < 0\) уравнение (2) приобретает вид

\[\hat{H}\psi_{k}(x) = E_{k}\psi_{k}(x)\]

Квантовое состояние, обладающее наименьшей энергией, называется
основным. Остальные состояния называются возбужденными состояниями.
Квантовые состояния дискретного спектра называют связанными состояниями.
Частица, находящаяся в связанном состоянии, не способна уйти на
бесконечность. То есть, плотность вероятности
\(|\psi_{k}(x)|^{2} \rightarrow 0\) при \(x \rightarrow \pm \infty\), но
на всех конечных расстояниях \(\neq 0\). В силу линейности стационарного
уравнения Шрёдингера, волновые функции математически определены с
точностью до постоянного множителя. Однако, из физических соображений,
волновые функции должны быть нормированы следующим образом:

\[\int\limits_{- \infty}^{+ \infty}|\psi_{k}(x)|^{2}dx = 1\]

Перейдем к случаю, когда \(E > 0\). Частица совершает инфинитное
движение. Оператор Гамильтона имеет непрерывный спектр собственных
значений. Квантовые состояния непрерывного спектра называют несвязанными
состояниями. Частица, находящаяся в несвязанном состоянии, способна уйти
в бесконечность.

\begin{figure}
\centering
\includegraphics[width=0.5\linewidth,height=\textheight,keepaspectratio]{potential_hole.png}
\caption{Потенциальное поле \(U(x)\)}
\end{figure}

Поскольку мы занимаемся численным моделированием квантовых состояний
только для дискретного спектра, небходима \emph{осцилляционная теорема}

\subsection{Осциляционная теорема}

Упорядочим собственные значения оператора Гамильтона в порядке
возрастания , нумеруя энергию основного состояния индексом «0»:
\(E_{0},E_{1},E_{2},\ldots,E_{k},\ldots\) . Тогда волновая функция
\(k\)-го состояния \(\psi_{k}(x)\) будет иметь \(k\) узлов (то есть
пересечений с осью абсцисс). Исключения: области, в которых
потенциальная функция бесконечна.

Волновые функции, представленные на рис. 2, хорошо иллюстрируют
осцилляционную теорему.

\begin{figure}
\centering
\pandocbounded{\includegraphics[keepaspectratio]{wave_functions.png}}
\caption{Прямоугольная потенциальная яма с бесконечными стенками:
волновые функции \(\psi_{1}(x),\psi_{2}(x)\) и \(\psi_{3}(x)\)}
\end{figure}

Для решения уравнения Шрёдингера удобно использовать атомные единицы
Хартри. В этих единицах уравнение (6), предполагая, что \(m = m_{e}\) ,
приобретает вид

\[\left\lbrack - \frac{1}{2}\frac{d^{2}}{dx^{2}} + U(x) \right\rbrack\psi(x) = E\psi(x)\]

Преобразуем (8) к форме

\[d^{2}\frac{\psi(x)}{dx^{2}} + q(E,x)\psi(x) = 0\], где

\[q(E,x) = 2\left\lbrack E - U(x) \right\rbrack\]

Формула для нахождения квантомеханического среднего ⟨x⟩ имеет следующий
вид:

\[\langle x\rangle = \int\limits_{a}^{b}\psi_{n}^{\ast (x)}x\psi(x)dx = \int\limits_{a}^{b}x\psi^{2}(x)dx\]

\section{Метод пристрелки. Алгоритм}

Решение стационарного уравнения Шрёдингера сводится к нахождению
собственных значений и собственных функций оператора Гамильтона, которые
будем вычислять по изложенному нижу алгоритму.

\ul{Алгоритм.}

Нас будут интересовать только основное и низковозбужденное состояния
электрона для потенциального профиля (1).

Поскольку для собственных значений известна оценка снизу (4), то удобно
начинать с вычисления энергии и волновой функции основного состояния.
Оценим грубо энергию основного состояния. Подставим значение этой
энергии в уравнение (8). Это уравнение теперь становится обыкновенным
дифференциальным уравнением 2-го порядка с граничными условиями
\(\psi(a) = \psi(b) = 0\). Вместо ОДУ 2-го порядка (8) рассматриваем
эквивалентную систему 2-х ОДУ 1-го порядка:

\[\begin{cases}
\frac{d\psi(x)}{dx} & = \psi_{1}(x) \\
\frac{\psi_{1}(x)}{dx} & = - q(E,x)\psi(x)
\end{cases}\]

Для нахождения точного собственного значения оператора Гамильтона
процедуры интегрирования «вперед» и «назад» должны приводить к
одинаковым результатам (с точностью до постоянного множителя и ошибок
округления). При неравенсте \(E\) собственному значению, эти процедуры
дают различные функции, отношение которых в различных узлах сетки не
равно постоянной величине. Для оценки близости \(E\) к собственному
значению будем вычислять разность производных волновых функций,
полученных интегрированием «вперед» и «назад», в некотором узле сетки
\(x_{m}\)

\[f(E) = \frac{d\psi_{> (x)}}{dx}{|_{x}}_{m} - \frac{d\psi_{<}(x)}{dx}{|_{x}}_{m}\]

Здесь \(\psi > (x)\) и \(\psi < (x)\) - волновые функции, полученные
интегрированием «вперед» и «назад» соответственно; xm - узел сшивки
производных. Перед вычислением (15) необходимо масштабировать функции
\(\psi > (x)\) и \(\psi < (x)\) так, чтобы
\(\psi > \left( x_{m} \right) = \psi < \left( x_{m} \right)\). Такое
масштабирование будем называть математической нормировкой. Вычисление
\(f\) позволяет организовать процедуру поиска собственного значения
методом бисекции (см. блок-схему на рис. 3). Выберем нулевое приближение
к энергии основного состояния следующим образом:
\(E(0) = U_{\min} + \delta\), где \(\delta\) малая величина
\((\delta > 0)\), и вычислим соответствующее \(f(0)\) по формуле (15).
Верхний индекс в скобках означает номер итерации. Далее, будем
увеличивать энергию с шагом \(\mathrm{\Delta}E\) до тех пор, пока
величины \(f(i)\) на двух соседних шагах \(i\) и \(i - 1\) не будут
иметь разные знаки. Если шаг \(\mathrm{\Delta}E\) был меньше, чем
разность энергий соседних уровней (собственных значе- ний) в актуальном
диапазоне изменения \(E\), то можно быть уверенным, что искомое
собственном значение \(\in \left\lbrack E(i - 1),E(i) \right\rbrack\).
Далее для уточнения собственного значения с наперед заданной точностью ϵ
используется метод бисекции.

Задача Коши для дифференциального уравнения (12) в программе реша- ется
с использованием пакетов numpy и scipy, в частности с использованием
функции scipy.integrate.odeint. В финальной части программы
осуществляется вывод графиков функций \(f(x)\), \(\psi(x)\),
\(\left| {\psi(x)} \right|^{2}\), потенциальной ямы \(U(x)\) согласно
варианту, найденных энергий \(E_{0},E_{1}(n = 1)\) и найденных
квантомеханических средних \(\langle x\rangle\),
\(\langle x^{2}\rangle\).

\begin{figure}
\centering
\includegraphics[width=0.5\linewidth,height=\textheight,keepaspectratio]{shooting_bs.png}
\caption{Блок-схема процедуры поиска собственного значения}
\end{figure}
\end{document}
