

%%%%%%%%%%%%%%%%%%%%%%%%%%%%%%%%%%%%%%%%%%%%%%%%%%%%%%%%%%%%%%%%%
%
% БЫСТРАЯ СБОРКА (Texmaker):
% PdfLaTeX + View PDF
%
\documentclass[russian]{article}
\usepackage[utf8]{inputenc}
\usepackage[T1]{fontenc}
\usepackage{babel}

\usepackage{pscyr} % улучшенный кириллический шрифт
% пакеты amssymb и amsmath рекомендуется подключать всегда,
% если в вашем тексте присутствуют сколько-нибудь сложные формулы
\usepackage{amssymb}
\usepackage{amsmath}
\usepackage{amsfonts} % подкл. дополнительных мат. шрифтов
\usepackage{latexsym} % некоторые редкие символы
% \usepackage[dvips]{graphicx} % импорт графических изображений в LaTeX
\usepackage[pdftex]{graphicx} % импорт графических изображений в PdfLaTeX
\usepackage{geometry} %
\usepackage{indentfirst} % отступ в первом абзаце
\usepackage{cite} % ссылки на литературу в виде [1-5,9]
% \usepackage{listings}
\usepackage{fancyvrb} % улучшенный вариант verbatim
%\usepackage{shortvrb) % сокращенный вариант verbatim
% \usepackage{verbatim}
\usepackage[large]{caption2}
\usepackage{url}

\pagestyle{myheadings} % верхний колонтитул (номер страницы)

\newcommand{\imsize}{0.45\columnwidth}
\newcommand{\bukva}[1]{\begin{tabular}[b]{c} (#1)\\ \\ \\ \end{tabular}}

%Изменение значения междустрочного интервала для всего документа
%\linespread{1.5}

\geometry{verbose,a4paper,tmargin=2.5cm,bmargin=2.5cm,lmargin=2.5cm,rmargin=1.0cm}

\begin{document}
\section{Введение}

В настоящее время современные исследования различных научных областей не
могут обходиться без вычислений. С развитием науки и техники
исследовательские вычисления стали обрабатывать огромные массивы данных.

Принято считать, что человеку намного проще воспринимать визуальные
образы, чем большие наборы числовых значений. В связи с этим современные
инструменты моделирования позволяют инженерам воплощать свои разработки
в виде различных графических моделей.

Как известно, любая сложная система не существует вне пространства и
времени. Система имеет свойство изменять свои значения в зависимости от
различных факторов, одним из которых является время. Сложность
моделирования систем, меняющих своё состояние в зависимости от времени
заключается в том, что графически невозможно отобразить эти изменения на
статичном кадре. Наиболее грамотным и элегантным решением возникшей
проблемы является создание анимаций.

Анимация поведения сложной системы позволяет визуально оценить изменения
состояния системы и уже на этапе просмотра анимации выявить интересующие
ученого закономерности.

Зачастую для создания крупных исследований ученые из разных областей
науки работают вместе. Создание анимации упрощает не только совместное
исследование, но и презентацию полученных результатов.

\section{Постановка задачи}

В связи с вышеописанными преимуществами использования анимации в задачах
компьютерного моделирования, была поставлена задача провести
исследование инструментов для создания качественной научной анимации на
основе полученных данных, в ходе моделирования процесса релаксации
многоатомной системы.

Для достижения поставленной цели группа студентов: Дудкин И.А., Старухин
Д.М. под руководством доцента/доктора физ.-мат. наук, профессора кафедры
математического и прикладного анализа ВГУ Тимошенко Ю.К. проведёт
исследовательскую работу по ознакомительному анализу программного
комплекса для графической визуализации данных DISLIN.

\section{Обзор информация о библиотеке DISLIN}

DISLIN - высокоуровневая библиотека для графической визуализации данных
в качестве кривых, диаграмм в полярной системе координат, столбчатых
диаграмм, круговых диаграмм, трехмерных графиков поверхностей, контуров
и карт {[}1{]}.

DISLIN задуман как мощный и простой в использовании программный пакет
для учёных и программистов. Для отображения желаемого графического
результата требуется всего несколько графических процедур с небольшим
списком параметров. Для создания индивидуально настраиваемых графиков
можно использовать широкий спектр процедур настройки параметров.

Программное обеспечение доступно для нескольких компиляторов C, C++,
Fortran 77 и Fortran 90/95 в операционных системах Unix, Linux, FreeBSD,
Windows, Mac OSX и MS-DOS. Программы DISLIN практически независимы от
системы, их исходный код можно переносить с одной операционной системы
на другую без каких-либо изменений.

В некоторых операционных системах DISLIN также поддерживает языки
программирования Perl, Python, Java, Ruby, TCL, Julia, FreeBASIC, Free
Pascal, Go, R и интерпретатор C/C++ Ch. Интерпретатор DISLIN (DISGCL)
доступен для всех поддерживаемых операционных систем.

Важной особенностью библиотеки является тот факт, что DISLIN является
библиотекой со свободно распространяющимся исходным кодом. Каждый может
без ограничений использовать все возможности библиотеки, а так же
принять участие в развитии проекта.

\section{Используемые технологии}

В настоящем исследовании для моделирования процесса релаксации
многоатомной системы и создания графической анимации будет
использоваться язык программирования Python {[}4{]} версии 3.13.7. Язык
программирования Python был представлен Гвидо ван Россумом 20 февраля
1991 года и по настоящий день развивается сообществом Python Software
Foundation {[}5{]}.

\section{Установка библиотеки DISLIN}

Перед началом использования библиотеки необходимо произвести настройку
операционной системы и установить библиотеку. Дальнейшие примеры
установки и настройки будут представлены для операционной системы Linux
{[}2{]}. Для использования современных версий библиотеки необходимо
убедиться в наличии используемого дистрибутива Linux библиотеки
OpenMotif {[}3{]}. OpenMotif - это свободно-распространяемая версия
библиотеки виджетов Motif от Open Group.

Для начала установки скачаем интересующий нас дистрибутив библиотеки
DISLIN. Воспользуемся утилитой wget для скачивания. Исполнив следующий
код, получим скачанный архив:

\begin{verbatim}
wget https://www.dislin.de/downloads/linux/i586_64/dislin-11.5.linux.i586_64.tar.gz 
\end{verbatim}

\section{Пример использования библиотеки DISLIN для создания
статического кадра}

\section{Пример создания анимации с использованием библиотеки DISLIN}

\section{Заключение}

\section{Литература}

\begin{enumerate}
\item
  Сайт библиотеки DISLIN - url: \url{https://www.dislin.de/}
\item
  Форум операционной системы Linux - url:
  \url{https://www.linux.org.ru/}
\item
  Сайт библиотеки OpenMotif - url:
  \url{http://www.opengroup.org/openmotif/}
\item
  Сайт языка программирования Python - url:
  \url{https://www.python.org/}
\item
  Python Software Foundation - url:
  \url{https://www.python.org/psf-landing/}
\end{enumerate}
\end{document}
