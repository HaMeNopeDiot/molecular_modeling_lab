\documentclass[8pt,xcolor=pdftex,dvipsnames,table]{beamer}
%\documentclass[fullscreen=true,bookmarks=false]{beamer}
%\usepackage{ucs}
\usepackage[utf8x]{inputenc}        % кодировка (обязательно!)
%\usepackage[T2A]{fontenc}
\usepackage[english,russian]{babel} % Система переносов (обязательно!)
\usepackage{xcolor}
\usetheme{Singapore}
\usepackage{animate}
\usecolortheme{beaver}

\begin{document}

% выводим заглавие
\begin{frame}
\transdissolve[duration=0.2]
\titlepage
\end{frame}

\begin{frame}
\frametitle{Описание}

Построим график синусоиды: 
$y(x)=sin(x)$

%\animategraphics[<options>]{<frame rate>}{<file basename>}{<first>}{<last>}
%\animategraphics[autoplay,loop, width = 0.6\textwidth]{5}{PIX/plot}{001}{040}

\end{frame}

\begin{frame}
\begin{animateinline}[autoplay,loop]{10}
\includegraphics[width=0.9\textwidth]{graph_0.pdf}
\newframe
\includegraphics[width=0.9\textwidth]{graph_1.pdf}
\newframe
\includegraphics[width=0.9\textwidth]{graph_2.pdf}
\newframe
\includegraphics[width=0.9\textwidth]{graph_3.pdf}
\newframe
\includegraphics[width=0.9\textwidth]{graph_4.pdf}
\newframe
\includegraphics[width=0.9\textwidth]{graph_5.pdf}
\newframe
\includegraphics[width=0.9\textwidth]{graph_6.pdf}
\newframe
\includegraphics[width=0.9\textwidth]{graph_7.pdf}
\newframe
\includegraphics[width=0.9\textwidth]{graph_8.pdf}
\newframe
\includegraphics[width=0.9\textwidth]{graph_9.pdf}
\newframe
\end{animateinline}

\end{frame}

%++++++++++++++++++++++++++++
\end{document}